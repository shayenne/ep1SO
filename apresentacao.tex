%%%%%%%%%%%%%%%%%%%%%%%%%%%%%%%%%%%%%%%%%
% Beamer Presentation
% LaTeX Template
% Version 1.0 (10/11/12)
%
% This template has been downloaded from:
% http://www.LaTeXTemplates.com
%
% License:
% CC BY-NC-SA 3.0 (http://creativecommons.org/licenses/by-nc-sa/3.0/)
%
%%%%%%%%%%%%%%%%%%%%%%%%%%%%%%%%%%%%%%%%%

%----------------------------------------------------------------------------------------
%	PACKAGES AND THEMES
%----------------------------------------------------------------------------------------

\documentclass{beamer}
\usepackage[utf8]{inputenc}
\usepackage[portuguese]{babel}
\usepackage{ragged2e}
\usepackage{url}
\mode<presentation> {

% The Beamer class comes with a number of default slide themes
% which change the colors and layouts of slides. Below this is a list
% of all the themes, uncomment each in turn to see what they look like.

%\usetheme{default}
%\usetheme{AnnArbor}
%\usetheme{Antibes}
%\usetheme{Bergen}
\usetheme{Berkeley}
%\usetheme{Berlin}
%\usetheme{Boadilla}
%\usetheme{CambridgeUS}
%\usetheme{Copenhagen}
%\usetheme{Darmstadt}
%\usetheme{Dresden}
%\usetheme{Frankfurt}
%\usetheme{Goettingen}
%\usetheme{Hannover}
%\usetheme{Ilmenau}
%\usetheme{JuanLesPins}
%\usetheme{Luebeck}
%\usetheme{Madrid}
%\usetheme{Malmoe}
%\usetheme{Marburg}
%\usetheme{Montpellier}
%\usetheme{PaloAlto}
%\usetheme{Pittsburgh}
%\usetheme{Rochester}
%\usetheme{Singapore}
%\usetheme{Szeged}
%\usetheme{Warsaw}

% As well as themes, the Beamer class has a number of color themes
% for any slide theme. Uncomment each of these in turn to see how it
% changes the colors of your current slide theme.

%\usecolortheme{albatross}
% Mesma cor do trabalho original
%\usecolortheme{beaver}
%\usecolortheme{beetle}
%\usecolortheme{crane}
%\usecolortheme{dolphin}
%\usecolortheme{dove}
%\usecolortheme{fly}
%\usecolortheme{lily}
%\usecolortheme{orchid}
%\usecolortheme{rose}
%\usecolortheme{seagull}
%\usecolortheme{seahorse}
\usecolortheme{whale}
%\usecolortheme{wolverine}

%\setbeamertemplate{footline} % To remove the footer line in all slides uncomment this line
\setbeamertemplate{footline}[page number] % To replace the footer line in all slides with a simple slide count uncomment this line

%\setbeamertemplate{navigation symbols}{} % To remove the navigation symbols from the bottom of all slides uncomment this line
}
\usepackage{caption}
\usepackage{graphicx} % Allows including images
\usepackage{booktabs} % Allows the use of \toprule, \midrule and \bottomrule in tables

%----------------------------------------------------------------------------------------
%	TITLE PAGE
%----------------------------------------------------------------------------------------

\title{EP1} % The short title appears at the bottom of every slide, the full title is only on the title page

\author{Florence Alyssa \and Shayenne Moura} % Your name
\institute[USP] % Your institution as it will appear on the bottom of every slide, may be shorthand to save space
{
Sistemas Operacionais
 \\ Bacharelado em Ciência da Computação% Your institution for the title page
\medskip
\textit{} % Your email address
}
\date{14 de setembro de 2015} % Date, can be changed to a custom date





\begin{document}

\begin{frame}
\titlepage % Print the title page as the first slide
\end{frame}

\begin{frame}
\frametitle{Sumário}
\tableofcontents
\end{frame}


%\begin{frame}
%\frametitle{Visão Geral} % Table of contents slide, comment this block out to remove it
%\tableofcontents % Throughout your presentation, if you choose to use \section{} and \subsection{} commands, these will automatically be printed on this slide as an overview of your presentation
%\end{frame}

%----------------------------------------------------------------------------------------
%	PRESENTATION SLIDES
%----------------------------------------------------------------------------------------

%------------------------------------------------
\section{Objetivo} 
%------------------------------------------------

\begin{frame}
\frametitle{Objetivo}
Implementar um shell simples, para permitir a interação do usuário com o sistema operacional, e um simulador de processos com diversos algoritmos de escalonamento para esses
processos. 
\end{frame}

\section{Arquitetura do shell}
\begin{frame}
\frametitle{Arquitetura do shell: ep1sh}

O shell (ep1sh) é basicamente um processo que 
recebe os comandos através do readline e cria um novo processo através de um \emph{fork()}.

Neste novo processo é executado o comando digitado no shell.

Alguns comandos são válidos em ep1sh, são eles:

\begin{itemize}
\item cd
\item pwd
\item ls
\item ./ep1 ``argumentos''
\item exit
\end{itemize}

O último comando finaliza a execução do ep1sh.

\end{frame}
%------------------------------------------------
%------------------------------------------------
\section{Escalonadores} 
%------------------------------------------------


\begin{frame}
\frametitle{Características comuns}

\justifying
\end{frame}

\subsection{FCFS}
\begin{frame}
\frametitle{First Come First Served}

\justifying
\end{frame}

%------------------------------------------------
\subsection{SJF} 
%------------------------------------------------

\begin{frame}
\frametitle{Shortest Job First}

\justifying
\end{frame}

\subsection{SRTN}
\begin{frame}
\frametitle{Shortest Remaining Time Next}

\justifying
\end{frame}

%------------------------------------------------
\subsection{RR} 
%------------------------------------------------

\begin{frame}
\frametitle{Round-Robin}

\justifying

\end{frame}

%------------------------------------------------
\subsection{Escalonamento com prioridade} 
%------------------------------------------------

\begin{frame}
\frametitle{Escalonamento com prioridade}

\justifying
\end{frame}

\subsection{Escalonamento em tempo real}
\begin{frame}
\frametitle{Escalonamento em tempo real com deadlines rígidos}


\justifying
\end{frame}
%------------------------------------------------
\section{Resultados} 
%------------------------------------------------

\begin{frame}
\frametitle{FCFS} 
\justifying
\end{frame}

\begin{frame}
\frametitle{SJF} 
\justifying
\end{frame}

\begin{frame}
\frametitle{SRTN} 
\justifying
\end{frame}

\begin{frame}
\frametitle{RR} 
\justifying
\end{frame}

\begin{frame}
\frametitle{Escalonamento com prioridade} 
\justifying
\end{frame}

\begin{frame}
\frametitle{Escalonamento em tempo real} 
\justifying
\end{frame}



\end{document}
